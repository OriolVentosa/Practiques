\documentclass{article}

\usepackage[utf8]{inputenc}
\usepackage[T1]{fontenc}
\usepackage[catalan]{babel}
\usepackage{amsmath, amssymb, amsthm}
\usepackage{graphicx}
\usepackage[colorlinks,linkcolor=blue,citecolor=blue,urlcolor=blue]{hyperref}

\renewcommand{\baselinestretch}{1.5}

\title{Pràctica 2: Zeros de funcions}
\author{Oriol Ventosa Altimira: 1457285}
\date{11 de març de 2018}
\begin{document}
	\maketitle
	
	\newpage

	\section{$|X|$}
	
			    \begin{figure}[h!]
			    	\begin{center}	
			    		\begin{tabular}{|c|c|c|c|}
			    			\hline $|X|$ & Legendre & Chebyshev & Trapezis* \\
			    			\hline 2 & 1.154700538379252 & 1.570796326794897 & 1 \\
			    			\hline 4 & 1.042534857261527 & 1.110720734539592 & 1 \\
			    			\hline 6 & 1.019894093560785 & 1.047197551196595 & 1 \\
			    			\hline 8 & 1.011528063414442 & 1.026172152976611& 1 \\
			    			\hline 10 & 1.007521690531119 & 1.016640738463052& 1 \\
			    			\hline 12 & 1.005294394550929 & 1.011515159927459 & 0.9999999999999998 \\
			    			\hline 
			    		\end{tabular}
			    	\end{center}
			    \end{figure}
	
	*En aquest cas la n dels trapezis no correspon al nombre de nodes com a Chebyshev i Legendre, sino que correspon al nombre de particions equidistants. Per obtenir resultats més significatius.
			    
	Observem que el valor absolut s'aproxima millor amb els nodes de Gauss-Legendre que amb Gauss-Chebyshev. 
	
	Això és degut a que al calcular la suma: $\sum f(x_i)a_i$, $f(x)=|x|$ per Legendre, mentre que per Chebyshev $f(x)=|x|\sqrt{1-x^2}$. De manera que es produiràn més error d'operacions per Chebyshev que per Legendre.
	
	L'aproximació trapezis donen un resultat exacte. 
		
	Això és perquè com trapezis fa una aproximació lineal de la funció i, si agafem un nombre parell de particions, un dels nodes caurà en el 0 i la resta estaràn repartits equidistantment en el interval assolint els extrems del interval. De manera que el reultat de l'aproximació lineal amb aquests valors serà exactament el valor absolut.
	
	\newpage
	
	\section{$\frac{e^{-x^2}}{\sqrt{1-x^2}}$}
	
	\begin{figure}[h!]
		\begin{center}	
			\begin{tabular}{|c|c|c|c|}
				\hline $\frac{e^{-x^2}}{\sqrt{1-x^2}}$ & Legendre & Chebyshev & Trapezis \\
				\hline 2 & 1.755136095633485 & 1.90547226473018 & 3.660243802385925 \\
				\hline 4 & 1.887429733737458 & 2.02581000359298 & 2.729403584695697 \\
				\hline 6 & 1.929062299374638 & 2.026436763135321 & 2.426317390201392 \\
				\hline 8 & 1.951603171842518 & 2.02643806549702 & 2.280527942170952 \\
				\hline 10 & 1.965709948752931 & 2.026438066948347 & 2.196561891849559 \\
				\hline 12 & 1.975356733965012 & 2.026438066949342 & 2.142840039465186 \\
				\hline 
			\end{tabular}
		\end{center}
	\end{figure} 
	
	Té una raó similar a la del apartat anterior, la funció $\frac{e^{-x^2}}{\sqrt{1-x^2}}$ té assimptotes horitzontals als extrems, i com que al calcular la suma $\sum f(x_i)a_i$ per Legendre s'usa aquesta mateixa, és produiràn errors d'operacions.
	
	En canvi per Chebyshev la funció que s'usa per la formula és $f(x)=e^{-x^2}$ que es molt menys errors als ser calculada, ja que no té problemes als extrems.
			
	En aquest cas els trapezis fallen ja que la funció no esta ben definida a $x=-1$ i $x=1$, on hi tenim una assimptota vertical, de manera que hem d'agafar com a extrems nombres propers a x=-1 i x=1 per calcular una aproximació amb aquest mètode.
			
	Si agafem nombre cada vegada més propers a x=-1 i x=1 com a extrems del nostre interval d'integració obtindrem errors de cancel·lació.
	
	\newpage
	
	\section{Sistema d'equacions}
	
	Per trobar els queficients ai hem d'imposar que la formula $\int_{-1}^{1} w(x)f(x)\simeq \sum_{i=1}^{n}a_if(x_i)$ sigui exacte per tot $f(x)=x^k$ on $k$ amb $k=0,1,...n-1$.
	
	De manera que ens queda aquest sistema d'equacions:
	
	$$\int_{-1}^{1}w(x)x^k=\sum_{i=1}^{n}a_i(x_i)^k$$
	
	És un sistema compatible ja matriu generada pels coeficients $a_i$ té determinant diferent a 0, ja que és el determinant de Vandermonde amb totes $x_i$ diferents dos a dos.
	
	Quan substituim el pes per $w(x)=1$, obtenim el sistema:
	
	$$\int_{-1}^{1}x^k=\sum_{i=1}^{n}a_i(x_i)^k$$
	
	Que és facil de solucionar, ja que només s'ha d'integrar un polinomi.
	
	De manera que calcular els pesos de Legendre és relativament fàcil.
	
	En canvi, quan substituim el pes per $w(x)=\frac{1}{\sqrt{1-x^2}}$ obtenim el sistema:
	
	$$\int_{-1}^{1}\frac{x^k}{\sqrt{1-x^2}}=\sum_{i=1}^{n}a_i(x_i)^k$$
	
	Aquesta integral és més dificil de solucionar.
	
	Al solucionar aquest sistema obtindriem el coeficients de Chebyshef.
	
\end{document}
